	% sredit sadrzaj - ovo ne lici ni na sto
% pogreske - slajd preusmjeravanje u datoteku - ide u /tmp/test
% nezgodno ispisuje \ u \shell{}, mozda verbatim

\documentclass{beamer}

\usepackage[english]{babel}
\usepackage[utf8]{inputenc}
\usepackage{listings}
\usepackage{datetime}
\usepackage{graphics}
\usepackage{fancybox}
\usepackage{color}
\usepackage[normalem]{ulem}
\usepackage{tikz}
\usepackage{verbatim}
\usetikzlibrary{shapes,arrows,positioning}
\usetheme{CambridgeUS}
\usecolortheme{seagull}
% Changing of bullet foreground color not possible if {itemize item}[ball]
\DefineNamedColor{named}{Purple}{cmyk}{0.52,0.97,0,0.55}
\setbeamertemplate{itemize item}[triangle]
\setbeamercolor{title}{fg=Purple}
\setbeamercolor{frametitle}{fg=Purple}
\setbeamercolor{itemize item}{fg=Purple}
\setbeamercolor{section number projected}{bg=Purple,fg=white}
\setbeamercolor{subsection number projected}{bg=Purple}

\renewcommand{\dateseparator}{.}
\newcommand{\todayiso}{\twodigit\day \dateseparator \twodigit\month \dateseparator \the\year}
\newcommand{\shell}[1]{\texttt{#1}}
\lstset{basicstyle=\ttfamily,
	showstringspaces=false,
	commentstyle=\color{red},
	keywordstyle=\color{blue}
}

\title{Osnove korištenja operacijskog sustava Linux}
\subtitle{05. Varijable i kontrola toka}
\author[Mateo Stjepanović]{Mateo Stjepanović\\{\small Nositelj: dr. sc. Stjepan Groš}}
\institute[FER]{Sveučilište u Zagrebu \\
				Fakultet elektrotehnike i računarstva}
				
\date{\todayiso}

\begin{document}
    %\beamerdefaultoverlayspecification{<+->}
{
\setbeamertemplate{headline}[] % still there but empty
\setbeamertemplate{footline}{}

\begin{frame}
\maketitle
\end{frame}
}

\begin{frame}
\frametitle{Sadržaj}
\tableofcontents
\end{frame}

\section{Varijable}
\begin{frame}[t]
\frametitle{Varijable}
\begin{itemize}
  \item Varijable se mogu koristiti kako u bilo kojem drugom jeziku.
  \item Sličnost s programskim jezikom Python.
  	\begin{itemize}
  		\item Nije potrebno deklarirati varijablu
  		\item Moguće u nju ubaciti većinu tipova
  	\end{itemize}
  \item Dvije vrste varijabli
  	\begin{itemize}
  		\item Globalne varijable (Vidljivo iz svih shell-ova)
  		\item Lokalne varijable (Vidljive samo u lokalnom shell-u)
  	\end{itemize}
  
  \item Mogućnosti rada s varijablama
  	\begin{itemize}
		\item Spremanje vrijednosti u varijablu
		\item Čitanje vrijednosti iz varijable
  	\end{itemize}
\end{itemize}
\end{frame}



\begin{frame}[t]
\frametitle{Primjer postavljanja i čitanja varijabli}
\begin{itemize}
	\item varijabla=vrijednost
		 \begin{itemize}
		 	\item hello="Hello World"
		 	\item num=2
		 \end{itemize}
	 \item varijabla=\$(\shell{naredba}) - preusmjeravanje izlaza naredbe u varijablu
	 \item \shell{export} varijabla - pretvaranje varijable u globalnu
	 \item \shell{echo} \$varijabla
	 	\begin{itemize}
	 		\item \shell{echo}  \$hello
	 		\item \shell{echo}  \$num
	 	\end{itemize}
 	\item lokalna varijabla unutar skripte - \shell{local}
 	
\end{itemize}

\end{frame}
\begin{lstlisting}[language=bash,caption={Primjer uporabe lokalne varijable}]
#!/bin/bash
HELLO=Hello 
function hello {
	local HELLO=World
	echo $HELLO
}
echo $HELLO
hello
echo $HELLO
\end{lstlisting}
\section{Testovi}
\begin{frame}
\frametitle{Testovi}
\begin{itemize}
	\item Prije ulaska u kontrolu toka potrebno je navesti testove
	\item Testovi su označeni s \shell{[ ]}
	\item Vraćaju 0 ili 1 ovisno o tome zadovoljava li test zadani uvjet
	\begin{itemize}
		\item \shell{[ 1 -eq 1 ]}
	\end{itemize}
	\item NB: potreban razmak između zagrada i uvijeta
	\item \shell{echo \$?} ispisuje rezultat testa
\end{itemize}
\end{frame}

\section{Operatori usporedbe}
\begin{frame}
\frametitle{Operatori usporedbe}
\begin{itemize}
	\item Operatori određuju uvjete kontrole toka
	\item Vrste operatora:
	\begin{itemize}
		\item Operatori integer-a
		\item Operatori string-a
		\item Operatori datoteka		
	\end{itemize}
\end{itemize}
\end{frame}

\begin{frame}
\frametitle{Operatori integer-a}
\begin{itemize}
	\item -lt (manje od)
	\item -gt (veće od)
	\item -le (manje od ili jednako)
	\item -ge (veće od ili jednako)
	\item -eq (jednako)
	\item -ne (različito)
\end{itemize}
\end{frame}

\begin{frame}
\frametitle{Operatori string-a}
\begin{itemize}
	\item \textless (manje od)
	\item \textgreater (veće od)
	\item = (jednako)
	\item != (različito)
	\item -n (vraća 0 ako niz ima duljinu)
	\item -z (vraća 0 ako je niz prazan)
\end{itemize}
\end{frame}

\begin{frame}
\frametitle{Operatori datoteka}
\begin{itemize}
	\item -e (datoteka postoji)
	\item -s (datoteka ima veličinu veću od 0)
	\item -d (datoteka je direktorij)
\end{itemize}
\end{frame}

\section{Kontrola toka}
\begin{frame}
\frametitle{Kontrola toka}
\begin{itemize}
	\item Sada se možemo upustiti u kontrolu toka
	\item Tri naredbe za kontrolu toka:
	\begin{itemize}
		\item \shell{if}
		\item \shell{else}
		\item \shell{elif}
	\end{itemize}
	\item Znajući gore navedene operatore i testove idemo prikazati primjere kontrole toka
\end{itemize}
\end{frame}
\begin{itemize}
	\item IF - struktura i primjer
\end{itemize}
	\begin{lstlisting}[language=bash]
		#! /bin/bash
		HELLO="hello"
		if [ $HELLO="hello" ]
		then
			echo "$HELLO world"
		fi
	\end{lstlisting}
\begin{itemize}
	\item ELSE - struktura i primjer
\end{itemize}
	
	\begin{lstlisting}[language=bash]
		#! /bin/bash
		if [ -e datoteka.txt ]
		then
			cat datoteka.txt
		else
			echo "Ne postoji datoteka"
		fi
	\end{lstlisting}
\begin{itemize}
	\item ELIF - struktura i primjer
\end{itemize}
	
	\begin{lstlisting}[language=bash]
		#! /bin/bash
		pom="cool"
		if [ pom="cool" ]
		then
			echo "Linux je cool"
		elif [ pom="Cool" ]
		then 
			echo "Linux je Cool"
		else
		then
			echo "Netocno"
		fi
	\end{lstlisting}

\begin{frame}
\frametitle{Literatura}
\begin{itemize}
	\item[]
	\item[] \small\url{http://blog.tgrrtt.com/bash-101}
	\item[] \small\url{http://linuxcommand.org/lc3_wss0110.php}
	\item[] \small\url{https://www.cyberciti.biz/faq/set-environment-variable-linux/}
	\item[] \small\url{https://ryanstutorials.net/bash-scripting-tutorial/bash-variables.php}
	\item[] \small\url{http://tldp.org/LDP/Bash-Beginners-Guide/html/sect_03_02.html} 
	\item[] \small\url{http://tldp.org/HOWTO/Bash-Prog-Intro-HOWTO-5.html}
\end{itemize}
\end{frame}
\end{document}
